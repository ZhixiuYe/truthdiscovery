\documentclass[../main.tex]{subfiles}
\begin{document}

In the previous chapters, we have motivated the need for a general theoretical
framework for truth discovery. To work towards actually constructing one, it is
necessary to set out exactly what such a framework will consist of, and what
features and properties are required for it to be useful.

The main goal of developing the framework is to set out rigorous definitions
for what truth discovery is, which allows the current situation to be modelled
whilst also permitting a more general view. The key definitions will therefore
be
\begin{itemize}
\item What is the `input' to truth discovery? The input has been described in
terms of sources, facts, objects and conflicting claims, but this needs to be
formulated mathematically.

\item What is the `output'? We have stated that the output is trust and belief
scores for sources and facts, according to the most common approach taken in
the literature. However our aim is to study truth discovery in full generality,
and not just the algorithms already in existence. Therefore a more general view
could be taken if desired, so long as it can still model existing algorithms.

\end{itemize}

With these definitions in place, a truth discovery algorithm is simply a
mapping from the space of inputs to outputs. This abstracts away the
\emph{process} of performing truth discovery so that `algorithm' is not the
correct term to use. We opt for \emph{truth discovery operator} to describe a
mapping from inputs to outputs.

There are several criteria against which to judge the usefulness of the
developed framework.
\begin{itemize}

\item \textbf{Ability to model existing approaches}: We aim to find a unified
framework that allow as many existing algorithms in the literature as possible
to be represented.

\item \textbf{Simplicity}: the key definitions should be easy to interpret, and
should relate to intuitive notions of truth discovery in a clear way.

\item \textbf{Flexibility}: we wish to prove properties of operators, compare
different operators, and develop axioms, so the framework should be easy and
flexible to work in.

\item \textbf{Generality}: the framework should be general and `unopinionated'
enough to be useful as foundations for future work, i.e. it should not rely on
specific ideas and approaches to performing truth discovery. It should also be
general in the sense of facilitating easy comparison between truth discovery
and related areas in the literature. This will allow ideas in these areas to be
applied to truth discovery, e.g. many axioms from social choice could be
translated to truth discovery.

\end{itemize}

Once the framework has been established, we aim to develop axioms for
operators. In line with axiomatic foundations for other problems, the axioms
should represent intuitively desirable properties that a `reasonable' operator
should satisfy. The power of the axiomatic approach is to then consider
multiple axioms together; the types of results attained include
\emph{impossibility results}, where it is proved that no
operator\footnotemark{} can satisfy a set of axioms, and \emph{representation
theorems}, where a set of sound and complete axioms are found for a particular
operator. For example, in the context of ranking systems, the authors in
\cite{altman_foundations} show that two seemingly complementary and desirable
axioms cannot be satisfied simultaneously, which has implications when deciding
which ranking system to use in practise.

\footnotetext{
    We write `operator' as a blanket term to refer to social choice functions,
    ranking systems, annotation aggregators etc.
}

Requirements for `good' axioms include having simple interpretations and
representing desirable properties in some way or another.

To justify the decisions made in developing the framework, it is worth giving
an overview of the existing formalisms for truth discovery.

\subsection{Existing Formalisms}

The definitions of input in the popular truth discovery literature
\cite{li_survey, gupta_han_survey, pasternack, galland, zhang_qi_tang} are
generally compatible with each other: we have a set of sources and facts and
objects\footnotemark{}, and sources provide (or \emph{claim}) a set of facts
for objects. Some definitions are not in exactly this form, but can clearly be
restated in this form without changing the essence of the approach.

\footnotetext{
    Note that the terminology is not uniform across the literature; e.g. the
    survey in \cite{gupta_han_survey} refers to sources as `providers', and the
    facts in \cite{pasternack} are referred to as `claims' (i.e. the word
    `claim' is used as a noun, whereas we use it as a verb).
}

For example, in \cite{pasternack} there is no concept of objects, but instead
of \emph{mutual exclusion sets} of facts which cannot simultaneously be true.
However the mutual exclusion sets themselves can be seen as the objects linking
related facts together.

In \cite{galland}, there is no concept of objects, and sources may make
\emph{negative claims} where they state a fact is \emph{false}. However one may
view the `facts' (in the sense of \cite{galland}) as objects, and `true' and
`false' as the only two facts associated with each object.

There is more variety in the definitions of output. The treatment of sources is
generally the same: each source is assigned a \emph{trust score}, (usually a
number in $[0, 1]$), but differences appear for the treatment of facts. The two
main ideas are to assign each fact a \emph{belief score}, or to select a single
`true' fact for each object. Another view, taken in \cite{li_conflicts} for
example, is to produce a single value for each object to represent the true
fact, but where the true value may not have been claimed by any sources (for
example, a weighted average could be applied when facts are numeric values, and
could lead to this situation).

Selecting a single fact for each object can be seen as a special case of
assigning each fact a score; e.g. the true facts receive a score of 1 and all
others receive 0.

\subsection{Overview of Approach}
\label{sec:theory_approach_overview}

With this background in mind, we give an overview and justification of the
approach before the formal definitions are given in section
\ref{sec:theory_framework}.

For input, a graph-theoretic representation is chosen. Nodes will be sources,
facts and objects. Edges between nodes represent the obvious relations: an edge
from a source to a fact represents the source claiming that fact, and an edge
from a fact to an object indicates the object the fact relates to. Setting this
up in graph theory allows for simple interpretation, and allows concepts in
graph theory to be usefully applied to describe properties of the input (for
example, the notion of \emph{connected components} is key in axiom
\ref{axiom:indep}).

Using a well established tool such as graph theory is hoped to also provide
flexibility for future refinements to consider more complex problems: notions
such as weighted edges, annotated nodes etc. could be used to conveniently
describe additional properties of the input.

Finally, a graph representation is already used in the related area of ranking
systems \cite{altman_foundations, altman_personalised}. Using a similar set up
allows comparison between the two areas.

For output, consider the two main ideas discussed above: assigning each fact a
score and selecting a single true fact for each object. Since the latter is a
special case of the former, the former is more suitable for a general theory of
truth discovery.

Note that assigning a numeric score to each source and fact in particular
induces an \emph{ranking} of the sources and facts. We argue that the essence
of truth discovery lies more in this induced ranking that the particular
numerical scores, and will therefore define the output of truth discovery as a
pair of rankings (precisely, \emph{total preorders}) on the set of sources and
facts.

Indeed, when applying truth discovery methods to determine which sources to
trust and which facts to believe, one is interested in which sources/facts are
\emph{more believable than others}, not in the particular numeric values
produced by an algorithm. Additionally, the numeric values produced often do
not have any semantic meaning \cite{pasternack}, which prevents inter-algorithm
comparison.  The induced rankings can therefore act as a bridge between results
from different algorithms.

A similar view is also taken in social choice and the axiomatic theory of
ranking systems, where rankings instead of numeric scores are the main objects
of interest. Taking the same approach for here highlights the similarities
between truth discovery and these areas, and allows concepts in these areas to
be carried over into truth discovery.

Nevertheless, to model in their entirety the algorithms that produce numeric
scores, it will be possible to define such operators in the framework as more
general objects, but restrict our attention mainly to the ranking-output
operators.

\end{document}
