\documentclass[../main.tex]{subfiles}
\begin{document}

\section{Proof of proposition \ref{prop:symm_iff_fact_source_object_symm}}
\begin{proof}

The `only if' direction is clear. For the converse, suppose $T$ is source, fact
and object symmetric, and let $N$, $N'=\pi(N)$ be equivalent networks. Define
\begin{equation*}
\pi_\S(x) = \begin{cases}
    \pi(x) & \text{ if } x \in \S \\
    x      & \text{ if } x \in \F \cup \O
\end{cases}
\end{equation*}
Define $\pi_\F$, $\pi_\O$ similarly, and set $\sigma = \pi_\S \circ \pi_\F
\circ \pi_\O$. Then for $s \in \S$,
\begin{align*}
\sigma(s) &= \pi_\S(\pi_\F(\pi_\O(s))) \\
          &= \pi_\S(\pi_\F(s)) \\
          &= \pi_\S(s) \\
          &= \pi(s)
\end{align*}
Similarly $\sigma(f)=\pi(f)$ and $\sigma(o)=\pi(o)$ for $f \in \F$ and $o \in
\O$. Hence $\sigma=\pi$.

Let $s_1, s_2 \in \S$. Since $\pi_\O$ only permutes objects, we may apply
object-symmetry to get
$$ s_1 \sle_N^T s_2 \iff \pi_\O(s_1) \sle_{\pi_\O(N)}^T \pi_\O(s_2) $$
Then since $\pi_\F$ only permutes facts and $\pi_\S$ only permutes sources, we
may successively apply fact and object symmetry to the right hand side to get
\begin{align*}
s_1 \sle_N^T s_2 & \iff \sigma(s_1) \sle_{\sigma(N)}^T \sigma(s_2) \\
                 & \iff \pi(s_1) \sle_{\pi(N)}^T \pi(s_2) \\
                 & \iff \pi(s_1) \sle_{N'}^T \pi(s_2)
\end{align*}
An identical argument for fact ranking gives $f_1 \fle_N^T f_2 \iff \pi(f_1)
\fle_{N'}^T f_2$. Hence $T$ is symmetric.

\end{proof}

\section{Proof of proposition \ref{prop:same_facts_ranked_equally}}
\begin{proof}

For the first claim, consider the permutation $\pi = (s_1, s_2)$. We claim that
$\pi(N) = N$. Let $E$ be the edges in $N$ and $\pi(E)$ be the edges in
$\pi(N)$. For any $(s, f) \in \S \times \F$ we have three cases:
\begin{enumerate}
\item $s \notin \{s_1, s_2\}$: in this case $(\pi(s), \pi(f)) = (s, f)$ and
$(s, f) \in E$ iff $(\pi(s), \pi(f)) \in \pi(E)$ by definition, so
$(s, f) \in E$ iff $(s, f) \in \pi(E)$.

\item $s=s_1$: Here we have $(s_1, f) \in E$ iff $(s_2, f) \in E$ by
hypothesis. This is equivalent to $(\pi(s_2), \pi(f)) \in \pi(E)$ by definition
of $\pi(E)$, which by definition of $\pi$ is equivalent to $(s_1, f) \in
\pi(E)$.

\item $s=s_2$: As above, $(s_2, f) \in E$ iff $(s_2, f) \in \pi(E)$
\end{enumerate}
Also, it is clear that $(f, o) \in E$ iff $(f, o) \in \pi(E)$ for $f \in \F$,
$o \in \O$. We have shown that $(x, y) \in E$ iff $(x, y) \in \pi(E)$, i.e.
$\pi(E) = E$ and thus $\pi(N) = N$. Applying source-symmetry, this gives
\begin{align*}
    s_1 \sle_N^T s_2 & \iff \pi(s_1) \sle_{\pi(N)}^T \pi(s_2) \\
                     & \iff s_2 \sle_N^T s_1
\end{align*}
i.e. $s_1 \seq_N^T s_2$.

For the second claim, consider $\sigma = (f_1, f_2)$ and apply a similar
argument to the above. Note that we require $f_1$ and $f_2$ to be associated
with the same object here so that both facts have the same incoming and
outgoing edges in $N$.

\end{proof}

\section{Proof of proposition \ref{prop:symm_and_dict}}
\begin{proof}

Suppose $T$ is source-symmetric and a dictatorship with dictator $s^*$. Let
$N=(V,E)$ be a truth discovery network.
\begin{enumerate}
\item
    Let $s_1, s_2 \in \S$. Without loss of generality $s_1 \sle_N^T s_2$, since
    $\sle_N^T$ is complete. Consider the permutation $\pi=(s_1, s^*)$.  We have
    $s_2 \sle_{\pi(N)}^T s^*$ by dictatorship in $\pi(N)$, which by symmetry
    means $\pi^{-1}(s_2) \sle_N^T \pi^{-1}(s^*)$, i.e. $s_2 \sle_N^T s_1$.
    Hence $s_1 \seq_N^T s_2$ as required.

\item
    Let $f_1, f_2 \in \F$ such that there is a source $s$ with $(s, f_1) \in E$.
    Set $\sigma=(s, s^*)$. Then $\sigma(E)$ contains $(\sigma(s), \sigma(f_1))
    = (s^*, f_1)$, so we have $\sigma(f_2) \fle_{\sigma(N)}^T f_1 =
    \sigma(f_1)$ since the facts claimed by $s^*$ rank above all others.
    Applying symmetry gives $f_2 \fle_N^T f_1$ as required.
\end{enumerate}

This implies there is no source-symmetric strict dictatorship. If $T$ were such
an operator then $T$ is also a non-strict dictatorship, so symmetry implies
all sources are ranked equally, but this contradicts $s \slt_N^T s^*$ for all
$s \ne s^*$.

\end{proof}

\section{Proof of proposition \ref{prop:non_dict_and_non_bin_gen_dict_indep}}
\begin{proof}

The operator $T$ defined in example \ref{example:symm_and_gen_dict} is not a
dictatorship but is a binary generalised dictatorship. Indeed, it is clearly a
binary generalised dictatorship from the definition. To show it is not a
dictatorship, it is sufficient by symmetry and proposition
\ref{prop:symm_and_dict} to find a network in which $T$ does not rank all
sources equally. A suitable example is a network $N$ where a source $s$ claims
two facts and all other sources claim only one fact. Then $Q_N = \{s\}$, so $s$
is ranked strictly above all other sources.

For an operator that is a dictatorship but not a binary generalised
dictatorship, fix two distinct sources $s^*, s^{**} \in \S$ and consider the
numerical operator $T'$ defined by
\begin{align*}
    t_N(s) & = \begin{cases}
        2 & \text{ if } s = s^* \\
        1 & \text{ if } s = s^{**} \\
        0 & \text{ otherwise}
    \end{cases} \\
    b_N(f) & = \begin{cases}
        1 & \text{ if } s^* \in \src(N, f) \\
        0 & \text{ otherwise}
    \end{cases} \\
\end{align*}
Clearly this is a dictatorship with dictator source $s^*$. Also, for any
network $N$ we have $s \slt_N^{T'} s^{**} \slt_N^{T'} s^*$; such a chain of
strict inequalities cannot occur for a binary generalised dictatorship, so we
are done.

\end{proof}

\section{Proof of proposition \ref{prop:unam_ground_indep}}
\begin{proof}

An operator that satisfies unanimity but not groundedness is $T$ that ranks
facts according to the function
$$
    r_N(f) = \begin{cases}
        1 & \text{ if } \src(N, f) \in \{\S, \emptyset\} \\
        0 & \text{ otherwise}
    \end{cases}
    \quad (f \in \F)
$$
i.e. $f_1 \fle_N^T f_2$ iff $r_N(f_1) \le r_N(f_2)$ (note that $T$'s ranking of
sources is irrelevant). Clearly $T$ satisfies unanimity but not groundedness:
consider any network $N$ in which there is a fact $f_1$ with no corresponding
sources, and a fact $f_2$ with at least one source, but whose sources are not
the whole of $\S$. Then $f_1 \not\fle_N^T f_2$ contrary to the requirements of
groundedness.

Reversing the fact ordering of $T$ gives an operator satisfying groundedness
but not unanimity.

\end{proof}

\section{Proof of lemma \ref{lemma:iterative_axiom_suff_conds}}

The following lemma will be used to prove the claim regarding weak
independence.

\begin{lemma}
\label{lemma:sequence_lemma}

Let $(a_n)_{n \in \Nat}$, $(b_n)_{n \in \Nat}$ be convergent sequences of real
numbers, and let $L = \lim_{n \rightarrow \infty}a_n$, $M = \lim_{n \rightarrow
\infty}b_n$.
\begin{enumerate}
    \item If $L < M$, then $a_n < b_n$ for sufficiently large $n$.
    \item If $a_n \le b_n$ for sufficiently large $n$, then $L \le M$.
\end{enumerate}
\end{lemma}

\begin{proof}

For the first claim, suppose $L < M$. Set $\epsilon = \frac{1}{2}(M - L) > 0$.
By definition of the limit $a_n \rightarrow L$, there is $N_1 \in \Nat$ such
that $|a_n - L| < \epsilon$ for all $n > N_1$. In particular, $a_n - L <
\epsilon = \frac{1}{2}M - \frac{1}{2}L$ and so $a_n < \frac{1}{2}(M + L)$.

On the other hand, by definition of $b_n \rightarrow M$ there is $N_2 \in \Nat$
such that $|b_n - M| < \epsilon$ for $n > N_2$, and in particular $b_n - M >
-\epsilon = \frac{1}{2}L - \frac{1}{2}M$ and so $b_n > \frac{1}{2}(M + L)$.
Thus, for $n > N:=\max\{N_1, N_2\}$ we have $a_n < \frac{1}{2}(M + L) < b_n$.

The second claim is now immediate; if $a_n \le b_n$ for sufficiently large $n$
and $L \le M$ were false, then by the first claim we would have $b_n < a_n$ for
sufficiently large $n$, which is a contradiction.

\end{proof}

\begin{proof}[Proof of lemma \ref{lemma:iterative_axiom_suff_conds}]
Let $I$ be as in the statement of the lemma.

\begin{enumerate}

\item This is immediate since $t_N^*(s)$ is a limit of numbers in $[0, 1]$
which is a closed set, so $t_N^*(s) \in [0, 1]$ (and similar for $b_N^*(f)$).

\item Let $N$ and $N' = \pi(N)$ be equivalent networks.  For any $s_1, s_2 \in
\S$ we have
\begin{align*}
    s_1 \sle_N^I s_2
        & \iff \lim_{n \rightarrow \infty}{t_N^n(s_1)} \le
            \lim_{n \rightarrow \infty}{t_N^n(s_2)} \\
        & \iff \lim_{n \rightarrow \infty}{t_{\pi(N)}^n(\pi(s_1))} \le
            \lim_{n \rightarrow \infty}{t_{\pi(N)}^n(\pi(s_2))} \\
        & \iff \pi(s_1) \sle_{\pi(N)}^I \pi(s_2) \\
        & \iff \pi(s_1) \sle_{N'}^I \pi(s_2)
\end{align*}
and for $f_1, f_2 \in \F$,
\begin{align*}
    f_1 \fle_N^I f_2
        & \iff \lim_{n \rightarrow \infty}{b_N^n(f_1)} \le
            \lim_{n \rightarrow \infty}{b_N^n(f_2)} \\
        & \iff \lim_{n \rightarrow \infty}{b_{\pi(n)}^n(\pi(f_1))} \le
            \lim_{n \rightarrow \infty}{b_{\pi(n)}^n(\pi(f_2))} \\
        & \iff \pi(f_1) \fle_{\pi(N)}^I \pi(f_2) \\
        & \iff \pi(f_1) \fle_{N'}^I \pi(f_2)
\end{align*}
Hence $I$ is symmetric.

\item Let $N$ be a truth discovery network and $f \in \F$ with $\src(N, f) =
\S$. By hypothesis there is $M \in \Nat$ such that $b_N^n(f) = 1$ for $n > M$,
so $b_N^*(f)=\lim_{n \rightarrow \infty}{1} = 1 \ge b_N^*(f')$ for any $f'$.
This means $f' \fle_N^I f$, i.e. $I$ satisfies unanimity.

\item Similarly, if $\src(N, f) = \emptyset$ then by hypothesis $b_N^n(f) = 0$
for $n$ sufficiently large, so $b_N^*(f)=0 \le b_N^*(f')$ for any $f'$, and $f
\fle_N^I f'$ as required for groundedness.

\item Let $G$, $N_1$, $N_2$ be as in the statement of the claim. For $s_1, s_2
\in G \cap \S$ we have
\begin{align*}
    s_1 \sle_{N_1}^I s_2
    & \iff \lim_{n \rightarrow \infty}{t_{N_1}^n(s_1)} \le \lim_{n \rightarrow
        \infty}{t_{N_1}^n(s_2)} \\
    & \iff \lim_{n \rightarrow \infty}{t_{N_2}^n(s_1)} \le \lim_{n \rightarrow
        \infty}{t_{N_2}^n(s_2)} \\
    & \iff s_1 \sle_{N_2}^I s_2
\end{align*}
and an identical argument proves the same for facts. Hence $I$ satisfies
independence of irrelevant stuff.

\item Let $G$, $N_1$, $N_2$ be as in the statement of the claim. Let $s_1, s_2
\in G \cap \S$ and suppose $s_1 \slt_{N_1}^I s_2$, i.e. $\lim_{n \rightarrow
\infty}{t_{N_1}^n(s_1)} < \lim_{n \rightarrow \infty}{t_{N_1}^n(s_2)}$. By
lemma \ref{lemma:sequence_lemma} part 1, there is $K \in \Nat$ such that
$t_{N_1}^n(s_1) < t_{N_1}^n(s_2)$ for $n > K$. For any such $n$, $\alpha_n \ge
0$ means $t_{N_2}^n(s_1) = \alpha_n \cdot t_{N_1}^n(s_1) \le \alpha_n \cdot
t_{N_1}^n(s_2) = t_{N_2}^n(s_2)$. Lemma \ref{lemma:sequence_lemma} part 2 then
gives $\lim_{n \rightarrow \infty}t_{N_2}^n(s_1) \le \lim_{n \rightarrow
\infty}t_{N_2}^n(s_2)$, i.e. $s_1 \sle_{N_2}^I s_2$.

An identical argument proves the result for fact ranking. Hence $I$ satisfies
weak independence.
\end{enumerate}

\end{proof}

\section{Proof of theorem \ref{theorem:sums_axioms}}

\begin{proof}
Assume Sums is convergent. Since trust and belief scores for Sums are always in
the range $[0, 1]$, lemma \ref{lemma:iterative_axiom_suff_conds} can be
applied.

\begin{enumerate}

\item\textbf{Symmetry:} We will use lemma
\ref{lemma:iterative_axiom_suff_conds} part 1. Let $N$ and $N'=\pi(N)$ be
equivalent networks. We show that $t_N^n(s) = t_{\pi(N)}^n(\pi(s))$ and
$b_N^n(f) = b_{\pi(N)}^n(\pi(f))$ for all $n \in \Nat$ by induction.

The base case $n=1$ is clear since $t_N^1$, $t_{\pi(N)}^1$, $b_N^1$ and
$b_{\pi(N)}^1$ are constant with value $\frac{1}{2}$. Suppose $n \in \Nat$ is
such that $t_N^n(s) = t_{\pi(N)}^n(\pi(s))$ and $b_N^n(f) =
b_{\pi(N)}^n(\pi(f))$ for all $s \in \S$ and $f \in \F$.

Let $s \in \S$. Note that $f \in \fact(N, s)$ iff $\pi(f) \in \fact(\pi(N),
\pi(s))$ by definition of $\pi(N)$. In particular, $\pi$ restricted to
$\fact(N, s)$ is a bijection into $\fact(\pi(N), \pi(s))$, so we may consider a
`substitution' $y = \pi(f)$ in the sum for $\hat{t}_N^{n+1}(s)$:

\begin{align*}
    \hat{t}_N^{n+1}(s) & = \sum_{f \in \fact(N, s)}{b_N^n(f)} \\
                       & = \sum_{y \in \fact(\pi(N), \pi(s))}{b_N^n(\pi^{-1}(y))} \\
                       & = \sum_{y \in \fact(\pi(N), \pi(s))}{b_{\pi(N)}^n(\pi(\pi^{-1}(y)))} \\
                       & = \sum_{y \in \fact(\pi(N), \pi(s))}{b_{\pi(N)}^n(y)} \\
                       & = \hat{t}_{\pi(N)}^{n+1}(\pi(s))
\end{align*}
Similarly, for $f\ \in \F$ note that $\pi$ restricted to $\src(N, f)$ is a
bijection into $\src(\pi(N), \pi(F))$: using the above and an identical
argument we get $\hat{b}_N^{n+1}(f) = \hat{b}_{\pi(N)}^{n+1}(\pi(f))$.

Now we have
\begin{align*}
    t_N^{n+1}(s)
    & = \frac{\hat{t}_N^{n + 1}(s)}{\max\limits_{x \in \S}{\hat{t}_N^{n + 1}(x)}} \\
    & = \frac{\hat{t}_{\pi(N)}^{n + 1}(\pi(s))}{\max\limits_{x \in
    \S}{\hat{t}_{\pi(N)}^{n + 1}(\pi(x))}} \\
\end{align*}

Note that $\pi$ restricted to $\S$ is a bijection into $\S$ itself, since
by definition of equivalent networks each of $\S$, $\F$ and $\O$ are closed
under $\pi$. Hence we may replace $\pi(x)$ in the maximum in the denominator
with simply $x$ by surjectivity of $\pi$, and so
$t_N^{n+1}(s)=t_{\pi(N)}^{n+1}(\pi(s))$. Similarly,
$b_N^{n+1}(f)=b_{\pi(N)}^{n+1}(\pi(f))$. Hence, by lemma
\ref{lemma:iterative_axiom_suff_conds}, Sums satisfies symmetry.

\item\textbf{Non-dictatorship:} We have shown that Sums satisfies symmetry. In
particular Sums satisfies source-symmetry, so given proposition
\ref{prop:symm_and_dict} it suffices to show that there is at least one
truth discovery network $N$ for which Sums does not rank all sources equally.
The network shown in figure \ref{img:sums_not_all_equal_trust} is a suitable
example. In this network there are three sources $A$, $B$ and $C$, and two
facts $D$ and $E$, each relating to a different object\footnotemark.

\footnotetext{
    If $\S$ contains more than three sources we consider all other sources to
    have no outgoing edges (and similarly for if $\F$ contains more than two
    facts). Note that the objects to not play any role in Sums.
}

We will show that Sums converges on this network, and that $B$ ranks strictly
beneath $A$.

For brevity write $a_n$ for $t_N^n(A)$, $\hat{a}_n$ for $\hat{t}_N^n(A)$
and similar for $B, C, D, E$. We claim that for all $n > 1$, $a_n = 1$, $b_n =
c_n = \frac{1}{2}$ and $d_n = e_n = 1$. By definition, for all $n > 1$:
\[
    \hat{a}_n  = d_{n-1} + e_{n-1}, \quad
    \hat{b}_n  = d_{n-1}, \quad
    \hat{c}_n  = e_{n-1}
\]
\[
    \hat{d}_n  = \hat{a}_n + \hat{b}_n, \quad
    \hat{e}_n  = \hat{a}_n + \hat{c}_n
\]
Recalling that $d_1 = e_1 = \frac{1}{2}$, taking $n=2$ we get
\[
    \hat{a}_2  = 1, \quad
    \hat{b}_2  = \frac{1}{2}, \quad
    \hat{c}_2  = \frac{1}{2}
\]
\[
    \hat{d}_2  = \frac{3}{2}, \quad
    \hat{e}_2  = \frac{3}{2}
\]
and so $a_2 = 1$, $b_2 = c_2 = \frac{1}{2}$ and $d_2 = e_2 = 1$. For $n = 3$ we
get
\[
    \hat{a}_3  = d_2 + e_2 = 2, \quad
    \hat{b}_3  = d_2 = 1, \quad
    \hat{c}_3  = e_2 = 1
\]
\[
    \hat{d}_3  = \hat{a}_3 + \hat{b}_3 = 3, \quad
    \hat{e}_3  = \hat{a}_3 + \hat{c}_3 = 3
\]
and so $a_3 = 1$, $b_3 = c_3 = \frac{1}{2}$ and $d_3 = e_3 = 1$, i.e. the
trust/belief scores are unchanged. Repeating in this way we see that $a_n = 1$
and $b_n = \frac{1}{2}$ for all $n > 1$. Hence $t_N^*(A) = 1 > \frac{1}{2} =
t_N^*(B)$: this means $B$ ranks strictly below $A$, which completes the proof.

\item\textbf{Unanimity}: Let $N$ be a truth discovery network and suppose $f \in \F$
is such that $\src(N, f) = \S$. Then, for $n > 1$ and any $f' \in \F$,
\[
    \hat{b}_n(f') = \sum_{s \in \src(N, f')}{\hat{t}_n(s)}
                  \le \sum_{s \in \S}{\hat{t}_n(s)}
                  = \hat{b}_n(f)
\]
using the fact that $\hat{t}_n(s) \ge 0$. Therefore $\max_{y \in
\F}{\hat{b}_n(y)} = \hat{b}_n(f)$, so $b_n(f) = 1$. By lemma
\ref{lemma:iterative_axiom_suff_conds} part 3, Sums satisfies unanimity.

\item\textbf{Groundedness}: Let $N$ be a truth discovery network and suppose $f
\in \F$ has $\src(N, f) = \emptyset$. By definition $b_n(f) = 0$ for all $n >
1$, so by lemma \ref{lemma:iterative_axiom_suff_conds}, Sums satisfies
groundedness.

\item\textbf{Weak independence}: To show weak independence we will use lemma
\ref{lemma:iterative_axiom_suff_conds} part 6. Let $N_1$, $N_2$ be
truth discovery networks with a common connected component $G$. Suitable
sequences $(\alpha_n)_{n \in \Nat}$, $(\beta_n)_{n \in \Nat}$ will be defined
recursively. For $n = 1$, set $\alpha_1 = \beta_1 = 1$. We have, by definition
of Sums,
\[
    t_{N_2}^1(s) = \frac{1}{2}
                 = \alpha_1 \cdot \frac{1}{2}
                 = \alpha_1 \cdot t_{N_1}^1(s)
\]
\[
    b_{N_2}^1(f) = \frac{1}{2}
                 = \beta_1 \cdot \frac{1}{2}
                 = \beta_1 \cdot b_{N_1}^1(f)
\]
for any $s \in G \cap \S$ and $f \in G \cap \F$. Now suppose $n \in \Nat$ is
such that there are $\alpha_{n - 1}, \beta_{n - 1}
\ge 0$ with
\[
    t_{N_2}^{n - 1}(s) = \alpha_{n - 1} \cdot t_{N_1}^{n - 1}(s)
\]
\[
    b_{N_2}^{n - 1}(f) = \beta_{n - 1} \cdot b_{N_1}^{n - 1}(f)
\]
for all $s \in G \cap \S$ and $f \in G \cap \F$. Fix a source $s \in G \cap
\S$. Let $E_1$, $E_2$ and $E_G$ denote the set of edges in $N_1$,$N_2$ and $G$
respectively. Note that for any fact $f \in \F$,
\begin{align*}
    f \in \fact(N_1, s) & \iff (s, f) \in E_1 \\
                        & \iff (s, f) \in E_G \\
                        & \iff (s, f) \in E_2 \\
                        & \iff f \in \fact(N_2, s)
\end{align*}
so $\fact(N_1, s) = \fact(N_2, s)$. Hence
\begin{align*}
    \hat{t}_{N_2}^n(s) & = \sum_{f \in \fact(N_2, s)}{b_{N_2}^{n - 1}(f)} \\
                       & = \sum_{f \in \fact(N_1, s)}{\beta_{n - 1} \cdot b_{N_1}^{n - 1}(f)} \\
                       & = \beta_{n - 1}\sum_{f \in \fact(N_1, s)}{b_{N_1}^{n - 1}(f)} \\
                       & = \beta_{n - 1}\hat{t}_{N_1}^n(s)
\end{align*}
Write
\[
    \gamma_1 = \frac{1}{\max\limits_{x \in \S}{\hat{t}_{N_1}^n(x)}},
    \quad
    \gamma_2 = \frac{1}{\max\limits_{x \in \S}{\hat{t}_{N_2}^n(x)}}
\]
so that
\begin{align*}
    t_{N_1}^n(s) & = \gamma_1 \cdot \hat{t}_{N_1}^n(s) \\
    t_{N_2}^n(s) & = \gamma_2 \cdot \hat{t}_{N_2}^n(s) \\
     & = \gamma_2 \cdot \beta_{n - 1} \cdot \hat{t}_{N_1}^n(s) \\
     & = \frac{\gamma_2 \cdot \beta_{n - 1}}{\gamma_1} \cdot \gamma_1 \cdot \hat{t}_{N_1}^n(s) \\
     & = \frac{\gamma_2 \cdot \beta_{n - 1}}{\gamma_1} \cdot t_{N_1}^n(s)
\end{align*}
Taking $\alpha_n = \frac{\gamma_2\beta_{n-1}}{\gamma_1}$ we have the desired
equality for trust scores. Note that $\gamma_1, \gamma_2 > 0$, so $\alpha_n \ge
0$.

The argument for belief scores is similar. Fix $f \in G \cap \F$. We have
$\src(N_1, f) = \src(N_2, f)$, so
\begin{align*}
    \hat{b}_{N_2}^n(f) & = \sum_{s \in \src(N_2, f)}{\hat{t}_{N_2}^n(s)} \\
       & = \sum_{s \in \src(N_1, f)}{\beta_{n-1} \cdot \hat{t}_{N_1}^n(s)} \\
       & = \beta_{n-1} \cdot \hat{b}_{N_1}^n(f)
\end{align*}
Write
\[
    \delta_1 = \frac{1}{\max\limits_{y \in \F}{\hat{b}_{N_1}^n(y)}},
    \quad
    \delta_2 = \frac{1}{\max\limits_{y \in \F}{\hat{b}_{N_2}^n(y)}}
\]
Then
\begin{align*}
    b_{N_1}^n(f) & = \delta_1 \cdot \hat{b}_{N_1}^n(f) \\
    b_{N_2}^n(f) & = \delta_2 \cdot \hat{b}_{N_2}^n(f) \\
     & = \delta_2 \cdot \beta_{n - 1} \cdot \hat{b}_{N_1}^n(f) \\
     & = \frac{\delta_2 \cdot \beta_{n - 1}}{\delta_1} \cdot \delta_1 \cdot \hat{b}_{N_1}^n(f) \\
     & = \frac{\delta_2 \cdot \beta_{n - 1}}{\delta_1} \cdot b_{N_1}^n(f)
\end{align*}
so we may take $\beta_n = \frac{\delta_2\beta_{n-1}}{\delta_1}$.

By induction, there exist sequences $(\alpha_n)_{n \in \Nat}$, $(\beta_n)_{n
\in \Nat}$, satisfying the hypothesis of lemma
\ref{lemma:iterative_axiom_suff_conds} part 6, so Sums satisfies weak
independece.

\end{enumerate}

\end{proof}

\section{Proof of theorem \ref{theorem:sums_non_indep}}

\begin{proof}
Let $N_1$ be the network shown in figure \ref{img:sums_not_all_equal_trust},
and $N_2$ be the network shown in figure \ref{img:sums_non_indep}. With
$G=N_1$, $G$ is a connected component of both networks. We have already shown
in the proof of theorem \ref{theorem:sums_axioms} that $t_{N_1}^*(A) = 1$ and
$t_{N_1}^*(B) = t_{N_1}^*(C) = \frac{1}{2}$, so in particular $A$ is ranked
strictly above $B$ in $N_1$. To prove Sums does not satisfy independence, we
will show that $A$ and $B$ are in fact ranked \emph{equally} in $N_2$; in
particular, we shall have $A \sle_{N_2}^{I_{sums}} B$ but $A
\not\sle_{N_1}^{I_{sums}} B$.

As before, write $a_n$ for $t_{N_2}^n(A)$ and similar for the other nodes. We
claim that, for $n > 1$:
\[
    a_n = \frac{2}{3^{n - 1}}, \quad
    b_n = c_n = \frac{1}{3^{n - 1}}, \quad
    s_n = t_n = u_n = 1
\]
\[
    d_n = e_n = \frac{1}{3^{n - 1}}, \quad
    v_n = w_n = x_n = 1
\]
The base case for induction is $n = 2$. We have
\[
    \hat{a}_2 = d_1 + e_1 = 1, \quad
    \hat{b}_2 = d_1 = \frac{1}{2}, \quad
    \hat{c}_2 = e_1 = \frac{1}{2}, \quad
\]
\[
    \hat{s}_2 = \hat{t}_2 = \hat{u}_2 = v_1 + w_1 + x_1
    = \frac{3}{2}
\]
\[
    \hat{d}_2 = \hat{a}_2 + \hat{b}_2 = \frac{3}{2}, \quad
    \hat{e}_2 = \hat{a}_2 + \hat{c}_2 = \frac{3}{2}
\]
\[
    \hat{v}_2 = \hat{w}_2 = \hat{x}_2 = \hat{s}_2 + \hat{t}_2 + \hat{u}_2
    = \frac{9}{2}
\]
The maximum trust score is $\frac{3}{2}$ and the maximum belief score is
$\frac{9}{2}$, so
\[
    a_2 = \frac{2}{3}, \quad
    b_2 = \frac{1}{3}, \quad
    c_2 = \frac{1}{3}, \quad
    s_2 = 1, \quad
    t_2 = 1, \quad
    u_2 = 1
\]
\[
    d_2 = \frac{1}{3}, \quad
    e_2 = \frac{1}{3}, \quad
    v_2 = w_2 = x_2 = 1
\]
Thus the claim holds for $n = 2$. Now suppose that the claim holds for the $(n
- 1)$-th iteration. We have
\[
    \hat{a}_n = d_{n-1} + e_{n-1}
              = \frac{2}{3^{n-2}}, \quad
    \hat{b}_n = d_{n-1}
              = \frac{1}{3^{n-2}}, \quad
    \hat{c}_n = e_{n-1}
              = \frac{1}{3^{n-2}}
\]
\[
    \hat{s}_n = \hat{t}_n = \hat{u}_n
    = v_{n-1} + w_{n-1} + x_{n-1}
    = 3
\]
\[
    \hat{d}_n = \hat{a}_n + \hat{b}_n
              = \frac{3}{3^{n - 2}}, \quad
    \hat{e}_n = \hat{a}_n + \hat{c}_n
              = \frac{3}{3^{n - 2}}
\]
\[
    \hat{v}_n = \hat{w}_n = \hat{x}_n = 9
\]
The maximum trust and belief scores are 3 and 9 respectively, so we get
\[
    a_n = \frac{2}{3^{n-2}}\cdot\frac{1}{3}
        = \frac{2}{3^{n-1}}, \quad
    b_n = \frac{1}{3^{n-2}}\cdot\frac{1}{3}
        = \frac{1}{3^{n-1}}, \quad
    c_n = \frac{1}{3^{n-2}}\cdot\frac{1}{3}
        = \frac{1}{3^{n-1}}, \quad
\]
\[
    s_n = t_n = u_n = 1
\]
\[
    d_n = \frac{3}{3^{n-2}}\cdot\frac{1}{9}
        = \frac{1}{3^{n-1}}, \quad
    e_n = \frac{3}{3^{n-2}}\cdot\frac{1}{9}
        = \frac{1}{3^{n-1}}, \quad
\]
\[
    v_n = w_n = x_n = 1
\]
as required.

Finally, this means
\[ t_{N_2}^*(A) = \lim_{n \rightarrow \infty}\frac{2}{3^{N-1}} = 0 \]
\[ t_{N_2}^*(B) = \lim_{n \rightarrow \infty}\frac{1}{3^{N-1}} = 0 \]
and $A \seq_{N_2}^{I_{sums}} B$, which completes the proof.

\end{proof}

\end{document}
